\chapter{Analyse du déroulement du stage}
Les retours de ce stage pour moi peuvent être classé en trois catégories : Technique, Culture Générale et Organisationel.

\section{Technique}
Le retour le plus évident de ce stage au niveau technique est l'experience acquise sur la librairie \textit{JavaFX}. Je ne la connaissais pas en arrivant à Charleston, et j'ai mis un peu de temps à assimiler les conceptes qui n'était pas dans \textit{Swing}, sa principale source d'inspiration. Ces améliorations et trouvailles, comme les \textit{Property}, le système de hierarchie des noeuds ou encore le FXML valent vraiment le coup d'être apprisent et améliorent considérablement la vie du developpeur. J'ai quelques reproches, rien de très grave, comme les bugs dues à la jeunesse de la librairie, la documentation parfois imprecise et pas des plus claires et la librairie de graphique assez limité\footnote{Des modifications peu orthodoxes nous ont éte necessaire pour permettre à l'utilisateur de se déplacer dans le schema}, mais dans l'ensemble c'est une excelente libraire que je re-utiliserai et que j'explorerai plus en détails.

Nous utilisons \textit{JavaFX 8}, qui est compatible uniquement avec \textit{Java 8}, la dernière mouture du langage. J'ai pu découvrir les nouveauté de cette nouvelle version, et plus particulièrement les fonctions lambdas. Une fonction lambda est une fonction anonyme, passé à une autre fonction. Imaginons une fonction qui trie des objets en fonction d'une regle particulière qui indique si l'objet est plus grand. Cette règle implémenté dans un objet comparateur est passé en paramètre de la fonction. Il faut donc créer cet objet comparateur à chaque fois que l'on appelle la fonction. La syntaxe pour faire ceci dans jusqu'a Java 7 était très lourde. On voyait plus que l'on créait un objet qu'une regle. Dans Java 8, on voit maintenant que l'on créé une règle. C'est très agréable!

Ce n'est ce pendant pas le concept le plus naturel dans le langage \textit{Java}. Java est basé sur le paradigme de Programation Orientée Objet. Les fonctions lambdas ont été créé pour le paradigme de Programation Fonctionnelle. Il y a des langages basés uniquement sur le principes de Programation Fonctionelle que j'ai découvers pendant ce stage, comme \textit{Haskell}. Approfondir ses connaissances dans ce paradigme est assez compliqué puisqu'il faut changer sa façon de penser du tout au tout, je n'ai donc pas eut vraiment le temps de me pencher dessus. Mais ce manque de temps n'est que temporaire.

J'ai joué avec le \textit{SVG}, je comprends un peu mieux comment ce format marche. J'ai aussi appris à utiliser Maven, un logiciel qui automatise la compilation de notre code. Souvent quand des entreprises compilent des logiciels, elles produisent plusieurs versions : debuggage, production, production pour un certain client ... Avant cela, elles testent le code qu'elle compile. Maven permet d'automatiser ces tests compilations. Il gêre aussi l'integration des librairies que l'on utilise directement. Ce n'est desormais plus au developpeur de réunir toutes les librairie dépendantes avant de compiler. Très pratique. 

Un outil qui m'a approrté des retours à la fois du côté technique et du côté organisationel : \textit{Git}, un outil de collaboratio  basé sur quelques principes simples. Cela n'empchêche pas ses possibilités d'être tellement étendues que cela le rend fascinant.J'avais déja quelques experiences avec \textit{Git} mais l'utiliser au quotidien m'a permis de vraiment me rendre compte de sa puissance. Un peu plus en particulier, j'ai appris à utiliser efficacement les branches, la difference entre pull et fetch, et différentes manières de communiquer et collaborer entre dépot.

\textit{GitHub} est un réseau social apportant ses fonctionnalité à internet et offrant d'autres services autour de Git. J'ai appris à communiquer et à collaborer sur ce site web. Je sais maintenant retrouver les différentes fonctionnalité de Git directement en ligne, je sais aussi utiliser leur \textit{Issues Tracker} et sais comment faire une pull request. Je fais maintenant énormement de choses avec \textit{Git} et \textit{Github}. Ce rapport de stage est par exemple entierement gêré avec \textit{Git} et disponible sur mon compte \textit{GitHub}.
Voyons comment Git a modelé notre façon de travailler en équipe.

%%Mentionner GitHub Pages (parce que serieux, quoi de mieux ?)

