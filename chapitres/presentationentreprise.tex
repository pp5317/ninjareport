\chapter{Présentation de l'entreprise}
\section{College of Charleston, l'organisation d'accueil}

Le "College of Charleston", abrégé \textit{Cofc}, fondé en 1770, situé dans la ville de Charleston, SC, USA, regroupe onze mille étudiants dans des programmes appartenant majoritairement au premier cycle d’études supérieur américain. Le champ des domaines enseignés est large, et va des arts aux sciences, en passant par le business, l'éducation ou encore les humanités.
 
Il est situé dans la ville de Charleston, ville de plus de 120 000 habitants situé sur la côte Atlantique des États-Unis, lieu touristique reconnu pour sa valeur historique, sa vie culturelle et sportive très active, un climat agréable et des conditions de vie plaisantes.
Une des particularités du Collège de Charleston est qu’il forme ses étudiants aux arts libéraux.  La notion des arts libéraux nous vient de l’antiquité. Le sens commun de cette expression désigne l'intégralité des matières qu’un homme doit étudier afin de devenir un citoyen accompli. Plus spécifiquement, \textit{a liberal arts college}a pour vocation de former ses étudiants à de multiples sujets, leur inculquant une culture générale solide, les incitant aux découvertes et à l’exploration.

Le \textit{Cofc} compte parmi ses diplômés des personnes ayant accomplis de grandes choses. Parmi eux, un acteur hollywoodien, un membre des New York Yankee, un musicien électronique reconnu, et un artiste ayant reçu un oscar pour son travail sur les effets spéciaux du film "Pirate des Caraïbe 3".

En plus d’enseigner, le collège a en son sein de nombreux laboratoires et participe activement à la recherche scientifique.

\section{Le laboratoire \textit{CIRDLES}, la cellule d’accueil}
J’ai effectué mon stage dans le laboratoire \textit{CIRDLES}, partie du Department of Computer Science, lui-même partie de la School of Sciences and Mathematics.

The School of Sciences and Mathematics est une des six divisions du Collège. Cette école est considérée comme la plus performante dans la recherche et l’enseignement depuis de nombreuses années au niveau de l’état de Caroline du Sud. Elle profite de son remarquable environnement pour sortir ses étudiants des salles de classes afin de les rapprocher de leurs sujets d’étude. Elle siège depuis 2011 dans le dernier ouvert par le \textit{Cofc}. Suivant la mission du Collège de Charleston de former des hommes curieux et explorateurs, elle propose en plus des programmes normaux, d’autres mélangeant plusieurs de ses disciplines.

Le département informatique propose quatre diplômes du premier cycle  qu’un diplôme du second cycle. Il est à dimension humaine, les professeurs connaissent et appellent leurs élèves par leur prénom, sont à l’écoute et accessibles aisément (Même spacialement parlant, leur bureaux se trouvant en face des salles de classes).
L’ambiance du département est exceptionnelle. Les étudiants se connaissent, s’entraident, se retrouvent dans une salle d’étude en deux parties, l'une consacrée à la détente et l’autre au travail. Ils s'occupent de plusieurs associations. Une des ces association, ACM, propose une conférence hebdomadaire donné par des professionnels de la sécurité informatique, une autre Gaming 101 organise une soirée jeux vidéos par semaine.

Le laboratoire CIRDLES développe des outils informatiques destinés aux géochronologistes du monde entier. Il est dirigé par Dr Jim Bowring. Le travail du laboratoire s’articule autour d’U-Pb-Redux et Tripoli, deux logiciels développés par Dr Bowring. Les membres du laboratoire, des étudiants rémunérés, se voient confier soit la réalisations de projets en rapport avec ces outils, comme la réalisation d’une applications Android permettant de consulter des fichiers issus d’U-Pb-Redux ou des tests, soit la réalisation de nouveaux outils plus généraux que ceux développés par le Docteur Bowring.

\section{Présentation des outils informatiques utilisés}

J’utilise un iMac doté deux écrans tournant sous Mac OSX 10.9.2. La librairie que l’on développe est codée avec le langage Java, dans sa huitième version. Elle étend les fonctionnalités de JavaFX, une librairie graphique permettant l’affichage par exemple de fenêtres.

Nous utiliserons aussi le site web Github pour nous organiser et mieux communiquer avec nos utilisateurs. Pour chaque projets qu'il héberge, GitHub propose un dépôt Git pour centraliser le code de l’application, un wiki pour la documentation et un système de suivi de bogue.
